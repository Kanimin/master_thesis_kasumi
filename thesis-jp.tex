%
%	$Id: template.tex,v 2.2 2019/07/26 02:31:32 kato Exp $
%
% KMD Thesis Template
%
% 以下の \documentclass の片方を、仕様する言語に合わせて選択する
% pLaTeX あるいは XeLaTeX で動作する筈。
%
\documentclass[12pt,a4j]{jreport}	% Japanese
%\documentclass[12pt,a4paper]{report}	% English

% 論文のスタイルをカンマで区切り、空白無しに指定する:
%    chicago: シカゴ形式にする
%    engineering: エンジニアリング形式にする (default)
%    doctor: 博士論文
%    proposal: 博士論文プロポーザル
%    master: 修士論文 (default)
%    final: 提出版
%    draft: ドラフト版 (提出日と前日は提出版になるので、通常は指定不要)
%    japanese: 日本語版を指定 (documentclass から判定するので通常は不要)
%    english: 英語版を指定  (documentclass から判定するので通常は不要)
%
\usepackage[master,engineering]{kmd-thesis}

%
% Overleaf で LaTeX を実行する場合、zip file を upload して Project
% を生成した後、右上の歯車をクリックして Settings 画面を呼び出し、
% 左下の Advanced Build Options の LaTeX Engine を ``LaTeX dvipdfmx''
% を選択しておくこと。
%

%%%%%%%%%% 各種変更
\makeatletter 
%
% subsection 表題に番号を振らない場合
%
%\renewcommand*{\l@subsection}{\@dottedtocline{2}{3.8em}{1.2em}}
%\renewcommand{\thesubsection}{\hskip-1.0em}

% 章見出しをヘッダに使う場合の文字の間隔を指定
% http://stackoverflow.com/a/3951837 から引用
%
%\newcommand{\addspaces}[1]{\@tfor\letter:=#1\do{\letter{\,}}}
   
\makeatother

% default の Latin font を変更する場合、いずれかを活かす:
%
%\renewcommand*\rmdefault{bch}   % Charter
%\renewcommand*\rmdefault{pnc}   % Century Schoolbook
 
%
% default の行間隔を調整するには、以下を活かす。Default は 1.2 だが、
% 和文論文でこれを 1.1 にするのは推奨しない
%
%\renewcommand{\baselinestretch}{1.1}

%%%%%%%%%%%%%% 論文用の情報 - 適宜編集されたい %%%%%%%%%%%%%%

%
% 年度、通常は現在の日付から自動的に計算されるので指定しなくてよい
%
% \syear{2018}

%
% 学生番号
%
\studentnumber{80733501}

%
% 日本語題目
%
\jtitle{遠隔地にいる人とのインターネットを介した楽器アンサンブルシステムの検討}

%
% 英語題目
%
\title{Practice of Design Thinking Workshop to Develop\\
  ``Media Innovator'' Leading Creative Society}

%
% 日本語副題 : 必要な場合のみ
%
%\jsubtitle{--- イノベーション世代のアプリケーション開発プラットホーム\\
%  環境からの視点に基づいて ---}

% 英語副題  : 必要な場合のみ
%
%\subtitle{--- In a Viewpoint of Application Development Platform Environments
%Innovation Generation ---}

%
% 日本語氏名 (姓と名の間に空白を入れて下さい)
%
\jauthor{冨安 香澄}

%
% 欧文氏名
%   (first name, last name の順に記入し、先頭文字のみを大文字にする。)
%
\author{Kasumi Tomiyasu}

%
% 指導教員:
%   (姓と名、名と称号の間に空白を入れて下さい)
%
\jaddadvisorymember{砂原 秀樹 教授}{(主指導教員)}
\jaddadvisorymember{南澤 孝太 教授}{(副指導教員)}

%
% 審査委員
%   (姓と名、名と称号の間に空白を入れて下さい)
%
% 最大6人まで指定できます。
%
\jaddthesismember{砂原 秀樹 教授}{(主査)}
\jaddthesismember{南澤 孝太 教授}{(副査)}
\jaddthesismember{ 准教授}{(副査)}


%
% キーワード5〜6個
%
\jkeywords{デザイン思考, 創造社会, ワークショップ, イノベーション, 教育}

%
% 5 or 6 Keywords
%
\keywords{Design Thinking, Creative Society, Workshop, Innovation, Education}
%

%
% 論文の投稿カテゴリーを一つ選ぶ
% (デザイン, サイエンス / エンジニアリング,
%   アクションリサーチ, 社会科学 / 人文科学)
%
\jcategory{デザイン}

%
% Choose one submission category from [Design, Science / Engineering,
% Social Science / Humanities, Action Research]
%
\category{Science / Engineering}

%
% 内容梗概
%   注: 行の先頭が\\で始まらないようにすること。
%
\jabstract{
\ruby{慶應義塾}{けいおうぎじゅく}大学大学院メディアデザイン研究科は、
創造社会にふさわしい産業を興しまた担う国際的に活躍する人材の育成を%
目的として設立された。
創造社会では、文化的社会的資源から価値を生み出す方法が変化するといわれている。
メディアデザイン研究科では、
創造性を重視した新しい国際社会を先導する人材をメディア・イノベータと呼び、
その育成に取り組んでいる。
加えて、社会と深く関わる実践的なプログラム「リアルプロジェクト」を導入し、
学生たちは自分とは異なる能力をもったメディア・イノベータとのコラボレーションを%
行いながらビジョン構成力と協調性を身につける。
このメディア・イノベータを育成する鍵となる手法がデザイン思考である。
本論文は、メディアデザイン研究科で独自に実践する%
デザイン思考ワークショップの手法とその成果を紹介する。
このワークショップは実際に研究科内で実施された他、
シンガポールやマカオといった海外の研究機関においても開催された。
これらの実践から得られた知見をまとめ、
来る創造社会を先導できるグローバルな視点を持ったメディア・イノベータを%
育成するためのデザイン思考ワークショップの方法を理論化する。
}

%
% Abstract of the thesis
%
\abstract{
In recent years, sensors are small enough and their prices are getting lower.
Sensors have change to be eccentrically located.
The Internet change transfer of large data.
Thereby, large scale sensor networks have been constructible.

In this kind of sensor network, we need to consider load balancing,
network redundancy and data reliability.
Using P2P technology is one of the solutions for load balancing and
network redundancy.

In this paper, we introduce how to ensure sensor data reliability on the system.
Approach way is detect fault and grant reliability metadata to the sensor data.
In fault detection, make a group by location and reliability.
This group is used for reduce the processing time.
User can use reliability point for select request data.
Experimental result shows this approach can grant reliability correctly over
90\%.
}

%
% 論文記述言語で謝辞を記述する (\jkmdacknowledgements は存在しない)
%
\kmdacknowledgements{
本研究の指導教員であり、
幅広い知見から的確な指導と暖かい励ましやご指摘をしていただきました%
慶應義塾大学大学院メディアデザイン研究科の砂原秀樹教授に心から感謝いたします。

研究の方向性について様々な助言や指導をいただきました慶應義塾大学大学院%
メディアデザイン研究科の加藤朗教授に心から感謝いたします。

研究指導や論文執筆など数多くの助言を賜りました慶應義塾大学大学院%
メディアデザイン研究科の杉浦一徳准教授に心から感謝いたします。

}

%%%%%%%%%%%%%%%%%%%%%%%%% document starts here %%%%%%%%%%%%%%%%%%%%%%%%%%%%

\begin{document}

\def\chaptermark#1{\markboth{#1}{ }}%
%
% 表紙 および アブストラクト
%
\titlepage
\comemberspage
\firstabstract
\secondabstract	% English Abstract
%
% 目次, 表目次、図目次
%
\toc
\ifPROPOSAL
   % Ph.D. Proposal do not require list of figures, list of tables, and
   % acknowledgements
\else
   \newpage
   \listoffigures
   \listoftables
\fi

\newpage
\pagenumbering{arabic}
\def\chaptermark#1{\markboth{\thechapter.\ #1}{ }}%

%%%%%%%%%%%%%%%%%%%%
%     論文本文     %
%%%%%%%%%%%%%%%%%%%%
%
% これ以降本文
%
% !TEX root = thesis-jp.tex

\chapter{序論}
\label{intro}

% Chicago Style のとき、フットノートをすべて章末に移動
\makeendnotes

慶應義塾大学大学院メディアデザイン研究科%
\footnote{\url{http://www.kmd.keio.ac.jp}} は、
創造社会にふさわしい産業を興しまた担う国際的に活躍する人材の育成を%
目的として設立された。
創造社会とは生産性や効率に代わって「創造性」が経済的価値を生み出す社会である。
そして創造性とは、新しいアイデア、表現、プロセスをゼロから生み出す能力である。
新しく創造されたものや活動は革新的な技術を生み出し活力のある経済基盤を創りだし、
心豊かな社会を構築し人の心を感動させる。
そして創造社会では個人の創造に価値があり、生活者が創造行為の主役になる。
個人個人が創造的になり、多様な価値観を認め合い、
個人の創造的な貢献によって集合的に大きな力を生み出すコラボレーション力が%
重要となる社会である。

創造社会では、文化的社会的資源から価値を生み出す方法が変化する。
高度なデジタル技術基盤によって地球規模で資源の再配分が行われ、
人間の創造性が回復する。
技術の進歩により、人々は日常生活で創造的な行為を活発に行うようになり、
クリエイティブな人たちが増える。
彼らの活動は新しい価値を生み出し、ネットワークを拡大して文化的社会的資源の%
再配分の障壁をなくし、自由な資源の共有を可能にする。
創造性が我々の日常生活において価値のある活動となり、人々は表現行為を楽しみます。
現在、新しい経済活動の登場を予感させる出来事が増えてきており、
多くの国では創造性を経済活動の資源として活用する%
クリエイティブ産業やカルチャー産業育成を国家戦略としている。

メディアデザイン研究科では、創造性を重視した新しい国際社会を先導する人材を%
メディア・イノベータと呼ぶ。
国際的な産学官連携によりメディア、コンテンツ、サービスを創出するプロジェクトを%
実践し、メディア・イノベータの育成に取り組んでいる。
先端的なコミュニケーション環境を駆使して国際的な多拠点連携ネットワークを構築し、
日吉をヘッドクオーターとし、大阪とシンガポールにサテライト拠点を整備している。
日本語と英語を公用語として使用しているため、世界各地から学生が集まり、
様々な経験や知識を有する教員とともに、多様性を大切にするコミュニティを形成し、
創造社会を先導する力を身につけるため学んでいる。

メディア・イノベータはデザイン創造性、テクノロジ創造性、マネジメント創造性、
そしてポリシー創造性の4つの分野を調和的に統合する。
さらに、各自が一つあるいは二つの高度に専門的な能力を身につけるように%
カリキュラムを構築している。
加えて、社会と深く関わるリアルプロジェクトと呼ぶ実践的なプログラムを導入し、
ビジョン構成力と協調性を育成するために自分とは異なる能力をもった%
メディア・イノベータとの過激なまでのコラボレーションを行っている。
デザイン、テクノロジ、マネジメント、
そしてポリシーの4つの創造性を連動させて多様な国際舞台でプロジェクトを推進する%
リーダーシップを発揮できる人材を育成することがメディアデザイン研究科の目標である。

デザイン、テクノロジ、マネジメント、そしてポリシーの4つの創造性を持った%
メディア・イノベータを育成する鍵となるのが%
デザイン思考\footnote{奥出直人 (2012) 『デザイン思考と経営戦略』,
  エヌティティ出版~\cite{Okude2012}
ではデザイン思考とはイノベーションデザインを行うための秘法であるとされている。}
である。

本論文は、メディアデザイン研究科で実際に実践している%
デザイン思考ワークショップについて述べるものである。
このワークショップは実際に研究科内の講義で開催された他、
シンガポールやマカオといった海外の研究機関においても実践しており、
来る創造社会を先導できるグローバルな視点を持った%
メディア・イノベータを育成することに寄与してきた。

% endnotes を置く
\putendnotes

% この下には何も書かない

% !TEX root = thesis-jp.tex

\chapter{関連研究}
\label{related}

% Chicago Style のとき、フットノートをすべて章末に移動するスクリプトです。
\makeendnotes
   
\section{創造社会に向けて}
\subsection{創造社会におけるコンセプトとは}

ダニエル・ピンクは著書 {\it A Whole New Mind: Why Right-Brainers Will
  Rule the Future: Riverhead Trade}
(邦訳『ハイ・コンセプト「新しいこと」を考え出す人の時代』)~\cite{Pink2006}
の中で、過去の歴史の中で価値の高いものは、
農業、工業、情報、コンセプトへと推移しており、
今日はコンセプトが世の中を動かしていると説く。
ピンクが述べるコンセプトとは日本語でいう「概念」という意味ではなく、
デザイン・ストーリー・調和・共感・遊び・生きがいの6つの感性を例として挙げる。
そして、コンセプトの源泉である右脳を発達させるための訓練について議論を展開させる。

We, then, ate your bread~\cite{Bellotti2008}.


\section{デザイン思考ワークショップ}
\subsection{IDEO}
We also ate rice~\cite{Bellotti2008} and bread~\cite{Sugiura2012,Uriu2012}.

\subsection{d-school}
We also ate rice~\cite{Bellotti2008} and bread~\cite{Tokuhisa2009}.

\subsection{SDM}

最近SDM\footnote{慶應義塾大学大学院システムデザイン・マネジメント研究科
  \url{http://www.sdm.keio.ac.jp/}}
もデザイン思考ワークショップを行なっている。
商業施設 iias Tsukuba(図\ref{iias})
{\footnote {iiasTsukuba \url{http://tsukuba.iias.jp}}} ではたぶん、
まだ行われていない。


%横文字にしたい場合(English only)
{\itshape Italics}

%写真の導入例
\begin{figure}[htbp]
\centering
  % if you have created iias.bb with "ebb iias.jpg" command in
  % figures directory, you don't have to specify the bb parameters
  % anymore.
  %\includegraphics[width=85mm, bb=0 0 640 480]{figures/iias.jpg}
\includegraphics[width=85mm]{figures/iias.jpg}
{\footnotesize(佐藤千尋博士論文~\cite{chihiro2014}より引用}
 \caption{Shopping center {\itshape iiasTsukuba}}
  \label{iias}
\end{figure}


%表の導入例
\begin{table}[ht]
\caption{街をぶらぶら歩く時の状態}
\label{cluster_category}
\centering
\small
\begin{tabu} to \linewidth {|c|c|c|c|}
\hline
% グリッドの大きさ指定
\makebox[8em][c]{暇な時} &
\makebox[8em][c]{駅に戻る時} &
\makebox[8em][c]{家に帰る時} &
\makebox[8em][c]{散策}\\
\hline
気分転換 & 考え事をしたい時 & 天気が良い時 & 買いものがしたい時\\
\hline
\end{tabu}
\end{table}

\section{DNS の性能}

DNS の性能はインターネット上でサービスを提供する際には重要なパラメータである。
その一つの方法は、DNSSEC で定義された NSEC/NSEC3 を拡張して利用することであり、
QNAME がキャッシュされた NSEC/NSEC3 にマッチする場合、
直ちに {\tt NXDOMAIN} を返答することである。
この拡張は RFC8198~\cite{rfc8198} で定義されている。

% endnotes を置く
\putendnotes

% この下には何も書かない


%\input{implementation-jp.tex}
%\input{evaluation-jp.tex}
%\input{conclusion-jp.tex}
%
%
%%%%%%%%%%%%%%%%%%%%%%%%%%%%%%%%%%%%%%%%%%%%%%%%%%%%%%%%%%%%%%%%%%%%
%%%%%%%%%%%%%%%%%%%%%%%%%%%%%%%%%%%%%%%%%%%%%%%%%%%%%%%%%%%%%%%%%%%%
%
% 本文の最後、これ以降は節番号をヘッダに載せない
%
\def\chaptermark#1{\markboth{#1}{ }}%

%
% 謝辞 : 日本語では参考文献の前
%
\acknowledgements
\acknowledgementstext

\newpage

%%%%%%%%%%%%%%%%
% Bibliography %
%%%%%%%%%%%%%%%%

%
% 参考文献
% ここでは \reference を使って、自分でリストを作るか、BibTeX を使って
% リストをつくって下さい。この例では BibTeX を作るような形式になってい
% ます。


%
% bibtexにリストされている、すべての文献を表示させたい場合は使用する。
% (本文中で引用されたものだけ使用する場合はこのまま)
%
% \ifDRAFT
%    \nocite{*}
% \fi

\bibliography{thesis-jp,thesis-en}

%%%%%%%%%%%%%%%%%%%%%%%%%
% Style of Bibliography %
%%%%%%%%%%%%%%%%%%%%%%%%%
%
\ifCHICAGO
   \bibliographystyle{jecon_edit}
\else
   \bibliographystyle{junsrt-url}
\fi
%
%%%%%%%%%%%%%%%%%%%%%%%
%
% もしあれば、付録
%
\appendix
\def\chaptermark#1{\markboth{\thechapter.\ #1}{ }}%
%
\input{appendix-jp}
%

\end{document}

% おしまい
