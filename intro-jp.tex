% !TEX root = thesis-jp.tex

\chapter{序論}
\label{intro}

% Chicago Style のとき、フットノートをすべて章末に移動
\makeendnotes

慶應義塾大学大学院メディアデザイン研究科%
\footnote{\url{http://www.kmd.keio.ac.jp}} は、
創造社会にふさわしい産業を興しまた担う国際的に活躍する人材の育成を%
目的として設立された。
創造社会とは生産性や効率に代わって「創造性」が経済的価値を生み出す社会である。
そして創造性とは、新しいアイデア、表現、プロセスをゼロから生み出す能力である。
新しく創造されたものや活動は革新的な技術を生み出し活力のある経済基盤を創りだし、
心豊かな社会を構築し人の心を感動させる。
そして創造社会では個人の創造に価値があり、生活者が創造行為の主役になる。
個人個人が創造的になり、多様な価値観を認め合い、
個人の創造的な貢献によって集合的に大きな力を生み出すコラボレーション力が%
重要となる社会である。

創造社会では、文化的社会的資源から価値を生み出す方法が変化する。
高度なデジタル技術基盤によって地球規模で資源の再配分が行われ、
人間の創造性が回復する。
技術の進歩により、人々は日常生活で創造的な行為を活発に行うようになり、
クリエイティブな人たちが増える。
彼らの活動は新しい価値を生み出し、ネットワークを拡大して文化的社会的資源の%
再配分の障壁をなくし、自由な資源の共有を可能にする。
創造性が我々の日常生活において価値のある活動となり、人々は表現行為を楽しみます。
現在、新しい経済活動の登場を予感させる出来事が増えてきており、
多くの国では創造性を経済活動の資源として活用する%
クリエイティブ産業やカルチャー産業育成を国家戦略としている。

メディアデザイン研究科では、創造性を重視した新しい国際社会を先導する人材を%
メディア・イノベータと呼ぶ。
国際的な産学官連携によりメディア、コンテンツ、サービスを創出するプロジェクトを%
実践し、メディア・イノベータの育成に取り組んでいる。
先端的なコミュニケーション環境を駆使して国際的な多拠点連携ネットワークを構築し、
日吉をヘッドクオーターとし、大阪とシンガポールにサテライト拠点を整備している。
日本語と英語を公用語として使用しているため、世界各地から学生が集まり、
様々な経験や知識を有する教員とともに、多様性を大切にするコミュニティを形成し、
創造社会を先導する力を身につけるため学んでいる。

メディア・イノベータはデザイン創造性、テクノロジ創造性、マネジメント創造性、
そしてポリシー創造性の4つの分野を調和的に統合する。
さらに、各自が一つあるいは二つの高度に専門的な能力を身につけるように%
カリキュラムを構築している。
加えて、社会と深く関わるリアルプロジェクトと呼ぶ実践的なプログラムを導入し、
ビジョン構成力と協調性を育成するために自分とは異なる能力をもった%
メディア・イノベータとの過激なまでのコラボレーションを行っている。
デザイン、テクノロジ、マネジメント、
そしてポリシーの4つの創造性を連動させて多様な国際舞台でプロジェクトを推進する%
リーダーシップを発揮できる人材を育成することがメディアデザイン研究科の目標である。

デザイン、テクノロジ、マネジメント、そしてポリシーの4つの創造性を持った%
メディア・イノベータを育成する鍵となるのが%
デザイン思考\footnote{奥出直人 (2012) 『デザイン思考と経営戦略』,
  エヌティティ出版~\cite{Okude2012}
ではデザイン思考とはイノベーションデザインを行うための秘法であるとされている。}
である。

本論文は、メディアデザイン研究科で実際に実践している%
デザイン思考ワークショップについて述べるものである。
このワークショップは実際に研究科内の講義で開催された他、
シンガポールやマカオといった海外の研究機関においても実践しており、
来る創造社会を先導できるグローバルな視点を持った%
メディア・イノベータを育成することに寄与してきた。

% endnotes を置く
\putendnotes

% この下には何も書かない
