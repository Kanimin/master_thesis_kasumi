%
%	$Id: template.tex,v 2.2 2019/07/26 02:31:32 kato Exp $
%
% KMD Thesis Template
%
% Pick ether \documentclass based on the language you use
% Should work with either pLaTeX or XeLaTeX
%
%\documentclass[12pt,a4j]{jreport}	% Japanese
\documentclass[12pt,a4paper]{report}	% English

% Specify your style in comma separated parameters (without space)
%    chicago: use Chicago sytle (otherwise Engineering style)
%    engineering: use Engineering style (default)
%    doctor: use Doctor style (otherwise Master style)
%    proposal: use Proposal style (implies doctor)
%    master: use Master style (default)
%    final: use Final style (otherwise Draft style)
%    draft: use Draft style (automatically becomes final mode
%           in the previous day and the day of submission dues.
%           so don't specify final or draft usually)
%    japanese: use Japanese version (automatically infer from the
%              documentclass style above, and usualy not necessary)
%    english: use English version (automatically infer from the
%             documentclass style above, and usualy not necessary)
%
\usepackage[master,engineering]{kmd-thesis}

%
% If you are going to run LaTeX process in Ovealeaf, you may need to
% upload the files to create a Project. Then, configure the project
% settings by click the gear symble at the top right. In Advanced Build
% Options (left bottom of Project Setting tab), choose ``LaTeX dvipdfmx''
% in LaTeX Engine.
%

%%%%%%%%%% Other tweaks
\makeatletter 
%
% Remove the number from subsection headings
%
%\renewcommand*{\l@subsection}{\@dottedtocline{2}{3.8em}{1.2em}}
%\renewcommand{\thesubsection}{\hskip-1.0em}

% Used to letterspace chapter titles
% adapted from http://stackoverflow.com/a/3951837
%
%\newcommand{\addspaces}[1]{\@tfor\letter:=#1\do{\letter{\,}}}
   
\makeatother

% Uncomment one of these to change the default Latin font:
%
%\renewcommand*\rmdefault{bch}   % Charter
%\renewcommand*\rmdefault{pnc}   % Century Schoolbook
 
%
% Uncomment to adjust the default line spacing.  The defaults are 1.2.
% Tighter spacing can be used, but this is not really recommended for
% Japanese text.
%
%\renewcommand{\baselinestretch}{1.1}

%%%%%%%%%%%%%% Thesis Parameters - Edit as Appropriate %%%%%%%%%%%%%%

%
% School Year, usually it automatically derived from current date
%
% \syear{2018}

%
% Student Number
%
\studentnumber{80733501}


%
% Thesis Title
%
\title{Practice of Design Thinking Workshop to Develop\\
  ``Media Innovator'' Leading Creative Society}

% English subtitle : only when necessary
%
%\subtitle{--- In a Viewpoint of Application Development Platform Environments
%Innovation Generation ---}


%
% Author Name
%    (specify as first name, last name, and capitalize the first letter)
%
\author{Masahiko Inakage}


%
% Research Advisors
%     First name comes first, and capitalize the top letter.
%     A white space required between first name and last name
%	If the title of your advisor is long, use two lines as following:
%		\addadvisorymember{Project Senior Assistant Professor}{}
%		\addadvisorymember{\ \ \ \ Marcos Sadao Maekawa}{(Co-Advisor)}
\ifDR
   % Ph.D Thesis
   \addadvisorymember{Professor Hideki Sunahara}{(Principal Advisor)}
   \addadvisorymember{Professor Akira Kato}{(Co-Advisor)}
   \addadvisorymember{Professor Kouta Minamizawa}{(Co-Advisor)}
\else
   % Master's Thesis
   \addadvisorymember{Professor Hideki Sunahara}{(Main Research Supervisor)}
   \addadvisorymember{Project Senior Assistant Professor}{}
   \addadvisorymember{\ \ \ \ Marcos Sadao Maekawa}{(Co-Advisor)}
\fi

%
% Thesis Review Committee
%     First name comes first, and capitalize the top letter.
%     A white space required between first name and last name
%     External Reviewer must be with his/her affiliation
%     When affiliation is too long, use the following format:
%        \addthesismember{Dr. Eiji Kawai}
%            {(Member, National Institute of Information \\
%             && \ \ and Communications Technology)}
%	If the title of your advisor is long, use two lines as following:
%		\addadvisorymember{Project Senior Assistant Professor}{}
%		\addadvisorymember{\ \ \ \ Marcos Sadao Maekawa}{(Co-Advisor)}
\ifDR
   \ifPROPOSAL
      % No thesis review committee for Ph.D. Proposal
   \else
   % Ph.D. Thesis
      \addthesismember{Professor Akira Kato}{(Chair)}
      \addthesismember{Professor Susumu Furukawa}{(Member)}
      \addthesismember{Professor Hiroyuki Kishi}{(Member)}
      \addthesismember{Associate Professor Akiko Orita}
                      {(Member, Kanto Gakuin University)}
   \fi
\else
   % Master's Thesis
   \addthesismember{Professor Hideki Sunahara}{(Chair)}
   \addthesismember{Professor Akira Kato}{(Co-Reviewer)}
   \addthesismember{Project Senior Assistant Professor}{}
   \addthesismember{\ \ \ \ MHD Yamen Saraiji}
                   {(Co-Reviewer)}
\fi


%
% 5 or 6 Keywords
%
\keywords{Design Thinking, Creative Society, Workshop, Innovation, Education}
%


%
% Choose one submission category from [Design, Science / Engineering,
% Social Science / Humanities, Action Research]
%
\category{Science / Engineering}


%
% Abstract of the thesis
%
\abstract{
In recent years, sensors are small enough and their prices are getting lower.
Sensors have change to be eccentrically located.
The Internet change transfer of large data.
Thereby, large scale sensor networks have been constructible.

In this kind of sensor network, we need to consider load balancing,
network redundancy and data reliability.
Using P2P technology is one of the solutions for load balancing and
network redundancy.

In this paper, we introduce how to ensure sensor data reliability on the system.
Approach way is detect fault and grant reliability metadata to the sensor data.
In fault detection, make a group by location and reliability.
This group is used for reduce the processing time.
User can use reliability point for select request data.
Experimental result shows this approach can grant reliability correctly over
90\%.
}

%
% Acknowledgements in Thesis Language (no \jkmdacknowlegements)
%
\kmdacknowledgements{
I am indebt to Professor Naohito Okude for guiding not only
about research but with many aspects of my life.
}

%%%%%%%%%%%%%%%%%%%%%%%%% document starts here %%%%%%%%%%%%%%%%%%%%%%%%%%%%

\begin{document}

\def\chaptermark#1{\markboth{#1}{ }}%
%
% Title Sheet and Abstract
%
\titlepage
\comemberspage
\firstabstract
%
% Table of Contents, List of Figures, List of Tables
%
\toc
\ifPROPOSAL
   % Ph.D. Proposal do not require list of figures, list of tables, and
   % acknowledgements
\else
   \newpage
   \listoffigures
   \listoftables
   %
   % In English, here comes acknowledgement
   %
   \acknowledgements
   \acknowledgementstext
\fi

\newpage
\pagenumbering{arabic}
\def\chaptermark#1{\markboth{\thechapter.\ #1}{ }}%

%%%%%%%%%%%%%%%%%%%%
% Main Thesis Text %
%%%%%%%%%%%%%%%%%%%%
%
% Here comes main text
%
% !TEX root = thesis-en.tex

\chapter{Introduction}
\label{intro}

% move notes to the last of the chapter in Chicago style
\makeendnotes

Keio University Graduate School of Media Design
(KMD)\footnote{\url{http://www.kmd.keio.ac.jp}} was established
to train talented individuals to work on the global stage building and
running new industries for the coming ``creative society,''
a world in which the driving force of the economy will be creativity
rather than productivity or efficiency.
``Creativity'' is the ability to produce new ideas, expressions,
and processes~\cite{Inakage2007}.
These new creations and the activities inspire give rise to
an economic base with the power and energy to bring forth innovative
technologies and enrich human societies.
The work of the individual is paramount in the creative society;
consumers lead creative activities.
Collaboration is all-important.
Individuals innovate, mutually recognizing a diversity of values
and making personal, imaginative contributions that collectively
result in extraordinary achievements and capacities.

The way in which we produce value from cultural and social resources
will change in the creative society.
Advanced digital technology provides a basis from which to redistribute
resources across the globe, and in doing so, restores the creativity of
human beings.
Technological progress produces value from creative activities
in everyday life and increases the number of creative individuals,
who produce value through their activities.
As networks expand, the barriers to redistribution of cultural and
social resources disappear, enabling resources to be shared freely.
Creativity will be a valuable activity within our day-to-day lives,
and people will fully embrace the ability to share and to express
themselves.
We are currently witnessing numerous events that hint
at the new economic activities to come.
Indeed, the fostering of the creative industries and cultural industries
that use creativity as a source of economic activity has,
in many countries, become a matter of national strategy.

At the Graduate School of Media Design, we call these future leaders
``media innovators.''
We implement projects that create media, content, and services
as international collaborations among industry, academia and government,
and in the process we train the media innovators of tomorrow.
The School features a state-of-the-art communications environment
and a broad, collaborative network
with some of the world's foremost universities.
The main campus is in Hiyoshi with satellites in Osaka and Singapore as well.
Both Japanese and English are our official languages and we welcome students
from around the globe.
Together with our highly experienced and insightful faculty members,
students learn the skills they need to lead the creative society
in this diverse community.

Media innovators harmonize and integrate skills in the four basic areas
of design creativity, technology creativity,
management creativity and policy creativity.
In addition, the curriculums are designed so that each student masters
advanced professional skills in one or two specialized areas.
It also features a practical program that we call ``Real Projects.''
These are intensely collaborative projects in which students work
with media innovators who possess different skills than they have to
address problems that impact their societies and communities,
and in the process, learn how to articulate a clear vision
and cooperate to solve the real world problems.
Our objective at the Graduate School of Media Design is to develop
leaders who will be able to combine the four creativities of design,
technology, management, and policy to drive projects on the global stage.

Through ``Real Projects'', KMD creates new media that go beyond
the parameters of conventional mass media.
This fusion of technology and media has the potential
to significantly reshape our everyday lives.
The content extends to physical artifacts and environment,
and it ultimately influences our social systems.
We learn and practice the five skills of fieldwork, strategic planning,
brainstorming, prototyping, execution, and verification as we develop,
verify, commercialize and establish companies and organizations
to exploit innovative content and technologies.
As we do so, we emphasize the global impact of our work in areas
such as standardization and institutional reform.

The Graduate School of Media Design looks forward to receiving applications
from potential students who are ambitious and passionate
about becoming media innovators and are eager to gain
the global perspective needed of the future leaders of the creative society.

% put endnotes here
\putendnotes

% Don't put anything below this line


% !TEX root = thesis-en.tex

\chapter{Related Works}
\label{related}

\ifCHICAGO 
   % move notes to the last of the chapter; engineering thesis doesn't this
   \makeendnotes 
\fi

\section{Shifting to Creative Society}
\subsection{A WHOLE NEW MIND}

Daniel Pink~\cite{Pink2006} said that the future belongs
to a different kind of person with a different kind of mind:
artists, inventors, storytellers-creative and holistic "right-brain" thinkers
whose abilities mark the fault line between who gets ahead and who doesn't.
Drawing on research from around the world,
Pink outlines the six fundamentally human abilities
that are absolute essentials for professional success and
personal fulfillment-and reveals how to master them.
A Whole New Mind takes readers to a daring new place,
and a provocative and necessary new way of thinking about a future
that's already here.
While, we ate your bread~\cite{Sugiura2012,Uriu2012}.

\section{Design Thinking Workshop}
\subsection{IDEO}
We also ate rice~\cite{Tokuhisa2009} and bread.

\subsection{d-school}
Also Pink~\cite{Pink2006} said something about creativity.


\subsection{SDM}
Recently SDM is also conducting design thinking workshop in
the Collaboration Complex\footnote{Graduate School of System Design
  and Management, Keio University \\\url{http://www.sdm.keio.ac.jp/}}.
But {\it iias Tsukuba} (Figure~\ref{fig:iias})~\footnote{{\it iias Tsukuba}
is one of largest shopping center located in Tsukuba city.
\url{http://tsukuba.iias.jp}}
has not conducted design thinking workshop yet. 


%inputting jpg
\begin{figure}[htbp]
\begin{center}
%  \includegraphics[width=85mm, bb=0 0 640 480]{figures/iias.jpg}
 \includegraphics[width=85mm]{figures/iias.jpg}\\
 {\scriptsize (Source: Ph.D. Thesis of Chihiro Sato~\cite{chihiro2014})}
\end{center}                 
  \caption{Shopping center {\itshape iiasTsukuba}}
  \label{fig:iias}
\end{figure}


%inputting tables
\begin{table}[ht]
\caption{state of your mind}
\label{cluster_category}
\centering
\small
\begin{tabu} to \linewidth {|c|c|c|c|}
\hline
% setting the size of the grid (only necessary if you want absolute column sizes)
\makebox[8em][c]{when} &
\makebox[8em][c]{where} &
\makebox[8em][c]{who} &
\makebox[8em][c]{what}\\
\hline
Bad & Good & Very Good & Good Good\\
\hline
\end{tabu}
\end{table}

\section{DNS Performance}

It is essential to improve DNS performance without chaning the semantics
of the DNS. One of the ways is to fully make use of DNSSEC
{\tt NSEC}/{\tt NSEC3} to allow full-service resolver
to respond {\tt NXDOMAIN} immedidately
when the QNAME matches with one of the cached {\tt NSEC}/{\tt NSEC3} records.
This modification has been proposed by RFC8198~\cite{rfc8198}.

% put endnotes here
\putendnotes

% Don't put anything below this line

%\input{implementation-jp.tex}
%\input{evaluation-jp.tex}
%\input{conclusion-jp.tex}
%
%
%%%%%%%%%%%%%%%%%%%%%%%%%%%%%%%%%%%%%%%%%%%%%%%%%%%%%%%%%%%%%%%%%%%%
%%%%%%%%%%%%%%%%%%%%%%%%%%%%%%%%%%%%%%%%%%%%%%%%%%%%%%%%%%%%%%%%%%%%
%
% Just chapter name at the end of main text
%
\def\chaptermark#1{\markboth{#1}{ }}%


\newpage

%%%%%%%%%%%%%%%%
% Bibliography %
%%%%%%%%%%%%%%%%

%
% Bibliography
% Using BibTeX (pbibtex) is suggested. You can also create the
% list of referances with \reference.
%

%
% Use to list all references given to bibtex.
% Don't modify if you need referenaces only cited in the text.
%
% \ifDRAFT
%    \nocite{*}
% \fi

\bibliography{thesis-en}

%%%%%%%%%%%%%%%%%%%%%%%%%
% Style of Bibliography %
%%%%%%%%%%%%%%%%%%%%%%%%%
%
\ifCHICAGO
   \bibliographystyle{econ_edit}
\else
   \bibliographystyle{unsrt-url}
\fi
%
%%%%%%%%%%%%%%%%%%%%%%%
%
% Appendix if any
%
\appendix
\def\chaptermark#1{\markboth{\thechapter.\ #1}{ }}%
%
% !TEX root = thesis-en.tex

\section{Example of Fault Detection Configuration File}

% move notes to the last of the chapter in Chicago style
\makeendnotes

Configuration File for Detecting Failures
{\small
\begin{verbatimtab}
<config>
<system>
<class>DefaultCompareClass</class>
</system>
<evaluate>
<compare_single_observation_point>
<function method="compareMax" recital="Temperature maximum threshold value"
type="Temperature"> <argument class="double">40.8</argument>
</function>
<function method="compareMin" recital="Temperature minimum threshold value"
type="Temperature"> <argument class="double">-41.0</argument>
</function>

<function method="compareChange" recital="Temperature change amount error"
type="Temperature"> <argument class="double">17.0</argument>
<argument class="int">1</argument>
</function>

<function method="compareConstant" recital="Temperature constant error"
type="Temperature"> <argument class="int">1</argument>
</function>
</compare_single_observation_point>

<compare_neighbor>
<function method="compareNeighbor" recital="Temperature neibor error"
type="Temperature"> <argument class="double">2.0</argument>
</function>
</compare_neighbor>

<compare_wide_area>
<function method="compareWide" recital="RainFall wide area error"
type="RainFall"> <argument class="double">10.0</argument>
</function>
</compare_wide_area>
</evaluate>
</config>
\end{verbatimtab}
}
%\begin{figure}[t]
%\begin{center}
%\includegraphics[width=120mm, height=100mm]{./figure/config.eps}
%\caption{障害検知設定ファイルの例}
%\label{appendix1}
%\end{center}
%\end{figure}

% put endnotes here
\putendnotes

% Don't put anything below this line

%

\end{document}

% End of File
