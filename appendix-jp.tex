% !TEX root = thesis-jp.tex

\section{障害検知設定ファイルの例}

% Chicago Style のとき、フットノートをすべて章末に移動するスクリプトです。
\makeendnotes

%障害検知の設定ファイルを以下に示す。
{\small
\begin{verbatimtab}
<config>
<system>
<class>DefaultCompareClass</class>
</system>
<evaluate>
<compare_single_observation_point>
<function method="compareMax" recital="Temperature maximum threshold value"
type="Temperature"> <argument class="double">40.8</argument>
</function>
<function method="compareMin" recital="Temperature minimum threshold value"
type="Temperature"> <argument class="double">-41.0</argument>
</function>

<function method="compareChange" recital="Temperature change amount error"
type="Temperature"> <argument class="double">17.0</argument>
<argument class="int">1</argument>
</function>

<function method="compareConstant" recital="Temperature constant error"
type="Temperature"> <argument class="int">1</argument>
</function>
</compare_single_observation_point>

<compare_neighbor>
<function method="compareNeighbor" recital="Temperature neibor error"
type="Temperature"> <argument class="double">2.0</argument>
</function>
</compare_neighbor>

<compare_wide_area>
<function method="compareWide" recital="RainFall wide area error"
type="RainFall"> <argument class="double">10.0</argument>
</function>
</compare_wide_area>
</evaluate>
</config>
\end{verbatimtab}
}
%\begin{figure}[t]
%\begin{center}
%\includegraphics[width=120mm, height=100mm]{./figure/config.eps}
%\caption{障害検知設定ファイルの例}
%\label{appendix1}
%\end{center}
%\end{figure}

% endnotes を置く
\putendnotes

% この下には何も書かない
